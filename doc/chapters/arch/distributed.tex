\section{Replication and Sharding}\label{sec:distributed}

The application uses \standout{replicas} to ensure high availability.

All programs that compound the application will always write to the primary
server (as enforced by \mongodb).

The scraper will use the \standout{read preference mode} \code{``primary''},
meaning that it will \emph{always} read from the primary server (if the primary
is not online, the read operation will fail). In fact the scraper needs to be
\emph{sure} that the configuration that reads from the database is the last
configuration submitted by the administrator. Moreover, since the objective of
the scraper is to download data and store it in the database, it would not make
sense to run the scraper when the primary server is not available (the scraper
would not be allowed to write to the database).

The server will use the \standout{read preference mode} \code{``nearest''},
meaning that it will read from the server with the lowest network latency. It is
in fact acceptable that some data that the server reads may not be up to date.

To ensure a good load balancing between cluster's servers the \code{AuthTokens}
and \code{MarketData} collections are \standout{sharded}. The former is queried
every time the server receives a request; The latter is the biggest collection
and it's intensively accessed when a strategy is executed.

The \code{\_id} field is the \standout{shard key} for the \code{AuthTokens}
collection. Since the values are random strings, the tokens will be sharded
randomly between the replica sets.

The \code{market} field is the \standout{shard key} for the \code{MarketData}
collection since it has a good granularity (so \mongodb{} can split the
documents into a large number of chunks). The field is hash-indexed, thus the
chunks will be randomly distributed across the configured servers. Moreover this
field will be always present in all queries to the database that get market
data, making the \mongodb's router to route each query directly to the machine
containing the requested data.

Note that, during the development of the application, we may decide to amend the
above architecture if other setups allows for better performances.
