\chapter{Application setup}\label{appendix:setup}

Server and Scraper have a set of command line options that allow to configure 
some connection parameters, furthermore the Server has an option for the
configuration of the directory where strategies will be saved. All the command
line options can be visualized with the \code{-h} parameter when starting the
application.

\begin{verbatim}
java -jar target/Server.jar -h
usage: Server [-h | --help] [-d <DBNAME> | --dbname <DBNAME>] [-c
              <CONNSTR> | --connection-string <CONNSTR>] [-D <DIR> |
              --strategies-dir <DIR>] [-P <PORT> | --scraper-port <PORT>]
              [-H <HOST> | --scraper-host <HOST>] [-p <PORT> | --port
              <PORT>] [-l <LEVEL> | --log-level <LEVEL>]
 -c,--connection-string <CONNSTR>   Set MongoDB connection string.
 -d,--dbname <DBNAME>               Set MongoDB database name.
 -D,--strategies-dir <DIR>          Set directory where strategies will be
                                    saved.
 -h,--help                          Print this message.
 -H,--scraper-host <HOST>           Set scraper hostname or ip address.
 -l,--log-level <LEVEL>             Set log level.
 -p,--port <PORT>                   Set listening port (default: 8888).
 -P,--scraper-port <PORT>           Set scraper connection port.
 -s,--standalone                    Disable MongoDB sharding.

LOG LEVELS:
ALL: print all logs.
FINEST: print all tracing logs.
FINER: print most tracing logs.
FINE: print some tracing logs.
CONFIG: print all config logs.
INFO: print all informational logs.
WARNING: print all warnings and errors. (default)
SEVERE: print only errors.
OFF: disable all logs.
\end{verbatim}


TODO server and scraper command line options; server and scraper run; cluster
run; standalone run; 
