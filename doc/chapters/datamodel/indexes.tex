\section{Indexes}\label{sec:indexes}

To ensure better performance of the application, some indexes are defined for
each collection.

Note that, during the development of the application, some of the following
indexes may be changed, removed, or other indexes may be added. Those decisions
will be guided by performance evaluations of the application on common
workloads.

\subsection{AuthTokens}

Auth tokens are always retrieved by the token hash, which is saved in the
\code{\_id} field that is already an unique index.

Additionally, the following \standout{TTL index} is defined:

\begin{lstlisting}[language=json]
{"expireTime": 1}, {expireAfterSeconds: 0}
\end{lstlisting}

This will instruct \mongodb{} to automatically remove tokens when they expire.

Note that in order to define a TTL index we can not save the auth tokens as
embedded documents inside the \code{Users} collection since indexes in
\mongodb{} are always \emph{collection-level} (so, setting a TTL index in an
embedded document, will result in the deletion of the entire root document when
the TTL expires).

\subsection{Sources}

We need to get a source, with all the markets, or a single market inside a
source. The following \standout{compound unique} index can support both queries:

\begin{lstlisting}[language=json]
{"_id": 1, "markets.id": 1}, {unique: true}
\end{lstlisting}

We may exploit the \mongodb's index intersection feature to support the query
that gets a single market, but since there is no need to get all the markets
with a specific \code{id} from all sources (we always get a specific market in a
specific source), we prefer to define a compound index in order to save
\mongodb{} the need to access two indexes to perform the query. Moreover, this
index enforces the uniqueness of the combination of the two fields.

\subsection{MarketData}

We need to get market data by \code{market} during the execution of a strategy.
We define an \standout{hashed index} on the \code{market} field:

\begin{lstlisting}[language=json]
{"market": "hashed"}
\end{lstlisting}

This is an hashed index since it will also be used as a shard key to shard the
collection over multiple servers, as defined in \secref{sec:distributed}.

Moreover, we need to sort documents based on the \code{start} field:

\begin{lstlisting}[language=json]
{"start": 1}
\end{lstlisting}

The \code{start} field is a redundancy since it has the same value of the
\code{t} field of the first candle in the document (\code{candles.0.t}). The
redundancy is added just to create this index to improve the sorting operation
(we could have set an index on \code{candles.t}, but this would have caused the
side effect of creating an index entry for each candle which is useless).

\subsection{Strategies}

Strategies are retrieved by file hash (the \code{\_id} field, already indexed)
or by name. Thus, we define the following \standout{unique index}:

\begin{lstlisting}[language=json]
{"name": 1}, {unique: true}
\end{lstlisting}

Since we also need to retrieve a specific run of a strategy, we also define the
following:

\begin{lstlisting}[language=json]
{"runs.id": 1}, {unique: true}
\end{lstlisting}

The uniqueness ensures that we can get a run using just its \code{id} and
without specifying which strategy it belongs to.

To allow the application to rapidly sort strategy's reports by the net profit
(commonly used to represent the overall performance of a strategy), we define
the following index:

\begin{lstlisting}[language=json]
{"runs.report.netProfit": 1}
\end{lstlisting}
