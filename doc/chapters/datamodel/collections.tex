\section{Collections}\label{sec:collections}

In this section we will define the database structure. The following collections
are defined:
\begin{enumerate*}[label=]
	\item \code{Users};
	\item \code{MarketData};
	\item \code{Sources};
	\item \code{Strategies}.
\end{enumerate*}

\subsection{Users}

\lstinputlisting[language=json, label={lst:userscollection},
caption={\code{User} document example.}]{users.json}

Each document represents a user.

Fields' description:
\begin{description}
	\item[\_id] \textit{(string)} Username;
	\item[password] \textit{(string)} Password hash;
	\item[is\_admin] \textit{(boolean, optional)} Specifies whether the user
		is the administrator or not; if not present, is considered
		\code{false};
	\item[auth\_token] \textit{(array, optional)} Array of embedded
		documents that represent the tokens used to authenticate the
		user with the server. This is an array since multiple clients
		can connect as the same user creating a session (a token) for
		each client (this allows for a session management feature to be
		implemented in future). The structure of these documents is the
		following:
		\begin{description}
			\item[token] \textit{(string)} Token hash (generated
				randomly);
			\item[expire\_time] \textit{(timestamp)} Token expire
				time.
		\end{description}
\end{description}

\subsection{MarketData}

\lstinputlisting[language=json, label={lst:marketdatacollection},
caption={\code{MarketData} document example.}]{marketdata.json}

Each document represents a month of a market.

Fields' description:
\begin{description}
	\item[\_id] \textit{(ObjectId)} An unique id automatically generated by
		the database;
	\item[market] \textit{(string)} Identifies the market. Format:
		\code{sourcename:market\-name};
	\item[month] \textit{(string)} Year and month of the trading days
		available in this document. Format: \code{YYYY-MM};
	\item[data] \textit{(array)} Array of embedded documents where each
		document represent a trading day. This array is pre-filled at
		the start of every month (it can not be empty). The structure of
		these documents is the following:
		\begin{description}
			\item[t] \textit{(timestamp)} The closing time of the
				day;
			\item[o] \textit{(numeric)} The opening price of the
				day;
			\item[h] \textit{(numeric)} The highest price of the
				day;
			\item[l] \textit{(numeric)} The lowest price of the day;
			\item[c] \textit{(numeric)} The closing price of the
				day;
			\item[v] \textit{(numeric)} The volume, \idest*{the
				amount traded during the day}.
		\end{description}
\end{description}

\subsection{Sources}

\lstinputlisting[language=json, label={lst:sourcescollection},
caption={\code{Sources} document example.}]{sources.json}

Each document represents a data source.

Fields' description:
\begin{description}
	\item[\_id] \textit{(string)} Source name;
	\item[enabled] \textit{(boolean, optional)} Specifies if users are
		allowed or not to test strategies on the markets of this data
		source. If not present, is considered \code{true};
	\item[streaming] \textit{(boolean, optional)} Specifies if the streaming
		from this data source is enabled or not. If not present, is
		considered \code{false}. If \code{enabled} is \code{false}, this
		is considered \code{false};
	\item[api\_key] \textit{(string, optional)} The API key used by the
		scraper to access the API endpoint. If not present, no API key
		is used;
	\item[markets] \textit{(array, optional)} array of embedded documents
		that lists the market available for this source. The structure
		of these documents is the following:
		\begin{description}
			\item[\_id] \textit{(string)} Name of the market;
			\item[granularity] \textit{(integer)} Minimum size,
				in minutes, of the trading day.
			\item[enabled] \textit{(boolean, optional)} Specifies if
				users are allowed or not to test strategies on
				this market. If not present, is considered
				\code{true}. If source's \code{enabled} is
				\code{false}, this is considered \code{false}.
		\end{description}
\end{description}

\subsection{Strategies}

\lstinputlisting[language=json, label={lst:strategiescollection},
caption={\code{Strategies} document example.}]{strategies.json}

Each document represents a strategy.

Fields' description:
\begin{description}
	\item[\_id] \textit{(string)} Hash of the file that defines the
		strategy;
	\item[runs] \textit{(array, optional)} Array of embedded documents that
		list all the tests made by the users with this strategy. The
		structure of these documents is the following:
		\begin{description}
			\item[config] \textit{(document)} Strategy's
				configuration used:
				\begin{description}
					\item[time]
						\textit{(timestamp)} Time of the
						last data on which the strategy
						has been executed;
					\item[market] \textit{(string)} Name of
						the market on which the strategy
						has been executed. Format:
						\enquote{sourcename:marketname};
					\item[granularity] \textit{(integer)}
						Trading day aggregation factor
						(this value is multiplied to the
						minimum granularity in order to
						obtain a higher level of
						aggregation);
					\item[\ldots] Other fields may be
						present (defined by the
						strategy).
				\end{description}
			\item[report] \textit{(document)} Report generated by
				the execution of the strategy:
				\begin{description}
					\item[net\_profit] \textit{(numeric)}
						The profit of the strategy,
						computed as the total profit
						minus the total loss of the
						trades (percentage of the
						initial amount);
					\item[gross\_profit] \textit{(numeric)}
						The total profit of the
						strategy, not counting the
						losing trades (percentage of the
						initial amount);
					\item[max\_drawback] \textit{(numeric)}
						The maximum drawback as defined
						in \chref{ch:specs}. This is
						expressed as a percentage of the
						initial amount;
					\item[hodl\_profit] \textit{(numeric)}
						The profit (\(\ge 1\)) or
						loss (\(< 1\)) of the Buy and
						Hold strategy (percentage of the
						initial amount).
					\item[trades] \textit{(array, optional)}
						Array of embedded documents that
						describes the transactions made
						by the strategy. The structure
						of these documents is the
						following:
						\begin{description}
							\item[entry\_time]
								\textit{(timestamp)}
								The time at
								which the trade
								is initiated;
							\item[exit\_time]
								\textit{(timestamp,
								optional)}
								The time at
								which the trade
								is concluded
								(not present if
								still open);
							\item[amount]
								\textit{(numeric)}
								The amount of
								asset traded.
								This is a value
								between 0 and 1
								(percentage of
								the initial
								amount);
							\item[profit]
								\textit{(numeric,
								optional)} The
								profit (\(\ge
								1\)) or loss
								(\(< 1\)) of the
								trade, expressed
								as a percentage
								of the traded
								amount (not
								present if the
								trade is still
								open).
						\end{description}
				\end{description}
		\end{description}
\end{description}
