\section{Administration manual}\label{sec:adminmanual}

The administrator has more options in the main menu:

\begin{verbatim}
Select an action:
1)      Browse strategies
2)      Upload a new strategy
3)      Browse users
4)      Create user
5)      Browse data sources
0)      Log-Out

Enter number:
\end{verbatim}

\subsection{Browse users}

When you select the ``Browse users'' menu entry, the application will ask for a
username. You can write it partially or leave the field empty to search for all
users. After that, you can select a user to \standout{delete} it.

\subsection{Browse data source}

When you select the ``Browse data sources'' menu entry, the application will
list all the data sources available in the system. You can select one of them in
order to change its configuration:

\begin{verbatim}
Select a Data Source:
1)      COINBASE
2)      BINANCE
0)      Go back

Enter number: 1

COINBASE | select an action:
1)      View details
2)      Disable Data Source
3)      Browse markets
4)      Delete data
0)      Go back

Enter number:
\end{verbatim}

You can \standout{disable} the source entirely (users will not be able to select
it anymore to run strategies); \standout{view and edit the markets} provided by
the data source; \standout{delete} the downloaded data for all the markets of
the data source.

\subsubsection{Browse markets}

When you select the ``Browse markets'' menu entry, the application will ask to
search by market name. Then you can select a market and edit its configuration:

\begin{verbatim}
COINBASE:BTC/EUR | Select an action:
1)      View details
2)      Change granularity
3)      Disable selectability
4)      Enable data sync
5)      Delete data
0)      Go back

Enter number:
\end{verbatim}

You can \standout{view the details} of the market; change the minimum
\standout{granularity} that the user can select when running a strategy on the
market; \standout{enable/disable} the market for the selection by the users
when running a strategy and the synchronization of data with the remote data
source; \standout{delete} the downloaded data of the market.

\subsubsection{Delete data}

When you select the ``Delete data'' menu entry, the application will ask for a
date. If you do not specify a date, all the data will be deleted. Otherwise,
only the data older that the specified date will be deleted. The date must be in
the format \code{yyyy-mm-dd hh:mm} \exgratia{\code{2017-03-22 12:15}}.
