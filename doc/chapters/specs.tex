\chapter{Specifications}\label{ch:specs}

The application developed for this workgroup task is a tool for
\standout{automated backtesting of financial trading strategies}.

The application allows the user to define a strategy and to test it with the
historical price action data of various markets. The user can then visualize
statistics on the performance of the strategy.

The \standout{User}, when logged-in, can write a new trading strategy. The
strategy will execute for every \emph{trading day}\footnote{A trading day is a
time span for which the price action, during that time span, is aggregated into
four values: highest, lowest, opening (the price at the start of the day) and
closing (the price at the end of the day) prices.}, having as input the highest,
lowest, opening and closing price for that day, and it must respond each time
with a buy/sell/do-nothing decision\footnote{More specifically, the strategy can
open multiple trades with different amounts and then close it at some time in
the future. A trade can be a long (buy) trade or a short (sell) trade. The close
operation, consist of making the inverse of the opening operation with the same
amount.}). Then, through the application, the user selects a market, the amount
to invest, the time interval of the trading day and others configurable
parameters for the selected strategy. After that, the application will apply the
defined strategy to all the historical data available for that market. At the
end, the application shows:
\begin{itemize}
	\item The net profit or loss;
	\item The gross profit \idest{the sum of the profits from all winning
		trades} and the gross lost;
	\item The Buy and Hold return (for comparison purpose);
	\item The total number of trades concluded by the strategy;
	\item The remaining open trades at the end of evaluation;
	\item The average amount of money involved in trades;
	\item The average duration of a trade (in terms of trading days);
	\item The percent of winning (profitable) and losing trades;
	\item The maximum number of consecutive losing trades;
	\item A graph of the evolution of the invested amount during time;
	\item The maximum drawback \idest{the maximum distance between a peak
		and the subsequent lowest trough in the graph above}.
\end{itemize}

For the development of strategies, the application provides a library for easily
compute common statistics over market data \exgratia{moving averages, standard
deviations, common indicators like the Relative Strength Index, Moving Average
Convergence/Divergence, Bollinger Bands margins, \etc}.

The application allows the user to display assets in different sort order (top
gainers, top losers, largest volume, \etc).

Once the strategy has been developed and tested it will be available to other users. This users will be able to visualize the reports for this strategy so that they can evaluate its goodness and improve it. 

The \standout{Administrator} can:
\begin{itemize}
	\item Add/remove users;
	\item Enable/disable data sources;
	\item Delete old or no-more-relevant data;
	\item Change various parameters for each data source \exgratia{API keys,
		data granularity, markets};
	\item Regenerate the data, re-downloading it from configured data
		sources;
	\item Enable/disable the download in streaming of new data;
	\item Enable/disable markets for selection by users;
	\item Set a rate-limit over the ability of users to test their
		strategies.
\end{itemize}

There's only one administrator. The administrator is created when the
application is run for the first time.

While the above specifications may be applied to any type of market, the
application will target only \standout{cryptocurrency markets} since they are
the easiest markets out there (always open markets; only buy/sell operations;
asset value depends only on price action).
