\section{Performance tests}\label{sec:performancetests}
To asses if the indexes designed effectively improve the performance of the
database we run some queries with the .explain("executionStats") method provided
by mongo shell. Results are the following:

\subsection{Sources}
On the collection Sources an index is present on the market.id field. In this
way the retrieval of informations on that market can be done using only this
field. This operation is widely used when retrieving the list of markets from a
given data source.

\lstinputlisting[language=json, label={lst:perfsourcesmarketid},
caption={Result execution stats on Sources.}]{sourcesmarket.id.json}

\subsection{MarketData}

The collection MarketData have two indexes, one is an hashed index on the market
field and another and is used to evenly balance documents among the cluster
members, the other is on the start field that can be useful when retrieving
strategies in order, for example during the execution of strategies.
Results of the effectiveness of our indexes are the following.

\lstinputlisting[language=json, label={lst:perfmarketdatamarket},
caption={Result execution stats on MarketData (market).}]{marketdatamarket.json}

\lstinputlisting[language=json, label={lst:perfmarketdatastart},
caption={Result execution stats on MarketData (market).}]{marketdatastart.json}


\subsection{Strategies}

The strategies collection has an index on the name field and other two on
runs.id and runs.report.netProfit field. The first index is useful when retrieving strategies
(i.e. Find Strategy use case), the others are useful when we want to retrieve a
report gven the id of that report. Last index is used when we want to sort the
strategies runs by netProfit. At the moment the application uses only the first
two
index, but the other has be mantained since future improvements may lead to an
implementation of a use case for this index, or can be used directly from the
shell to perform some analysis from the administrator.

\lstinputlisting[language=json, label={lst:perfstrategiesname},
caption={Result execution stats on Strategies (name).}]{strategiesname.json}


\lstinputlisting[language=json, label={lst:perfstrategiesrunsid},
caption={Result execution stats on Strategies (runs.id).}]{strategiesruns.id.json}



