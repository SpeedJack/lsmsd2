\section{Performance tests}\label{sec:performancetests}
To asses if the indexes designed effectively improve the performance of the
database we run some queries with the .explain("executionStats") method provided
by mongo shell. Results are the following:

\subsection{Sources}
On the collection Sources an index is present on the market.id field. In this
way the retrieval of informations on that market can be done withou the need of
the client to know other then that field.
field. This operation is frequently used when retrieving the list of markets from a
given data source. The benefits of having an index on that field are clear,
since each DataSource can have thousand of Markets inside with this index we can
retrieve informations on that market without searching throut searching through
the whole list as we can see from the \code{executionStats.totalKeysExamined}
field and \code{executionStats.totalDocsExamined} field (equal to 1).

\lstinputlisting[language=json, label={lst:perfsourcesmarketid},
caption={Result execution stats on Sources.}]{sourcesmarket.id.json}

\subsection{MarketData}

The collection MarketData have two indexes, one is an hashed index on the market
field, used to evenly balance documents among the cluster
members, the other is on the start field that can be useful during the execution
of strategies, to retrieve data in order.
Results of the effectiveness of our indexes are the following.
Since the hashed index is on the Market field, we can retrieve all the data for
a given market searching in a single shard. In this way, sorting can be done
directly in the shard, without the need to wait for all the data to arrive in a
single point to be reordered. In this way we can achieve filtering and sorting
operation, eploiting two indexes instead of one, with the advantage of having a
reduced communication overhead (one shard involved in the communication) and the
data arrive to the mongo router already sorted.
\lstinputlisting[language=json, label={lst:perfmarketdatamarket},
caption={Result execution stats on MarketData (market).}]{marketdatamarket.json}

\lstinputlisting[language=json, label={lst:perfmarketdatastart},
caption={Result execution stats on MarketData (market).}]{marketdatastart.json}


\subsection{Strategies}

The strategies collection has an index on the name field and other two on
runs.id and runs.report.netProfit field. The first index is useful when retrieving strategies
(i.e. Find Strategy use case), the others are useful when we want to retrieve a
report gven the id of that report. Last index is used when we want to sort the
strategies runs by netProfit. At the moment the application uses only the first
two
indexes, but the other has be mantained since future improvements may lead to an
implementation of a use case for this index, or can be used directly from the
shell to perform some analysis from the administrator.
The index on the strategy name has a clear impact on the Browse Strategies use
case, where users can set the name of the strategies. In this way the user can 
retrieve a Strategy without waiting for the database to look at every document
in the Strategies collection, as for the index on Market.id on MarketData
results can be asssessed looking at the \code{executionStats.totalKeysExamined}
field and \code{executionStats.totalDocsExamined} field.
The second index is similar but is on a nested array of documents, and is used when we
want to retrieve a report, after the View Strategy use case. In this way since a
Strategy may have thousands of runs, and so thousands of reports, we can achieve
a direct lookup instead of looking to all the reports for a given strategy.
Another advantage is that just one field is needed from the client to retrieve
the report, since the id field on runs is unique.

\lstinputlisting[language=json, label={lst:perfstrategiesname},
caption={Result execution stats on Strategies (name).}]{strategiesname.json}


\lstinputlisting[language=json, label={lst:perfstrategiesrunsid},
caption={Result execution stats on Strategies (runs.id).}]{strategiesruns.id.json}



